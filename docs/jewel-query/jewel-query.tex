\documentclass[a4paper,12pt]{article}
\title{Jewel Query Problem}
\date{\today}
\usepackage{listings}
\usepackage{graphicx}
\usepackage{color}
\usepackage{multicol}
\usepackage{fullpage}
\usepackage{amsmath}
\usepackage{amssymb}
\usepackage{lazylist}
\usepackage{synttree}
\usepackage{algorithm}

\newcommand{\TR}[1]{#1^{\rm T}}
\newcommand{\lproof}{$\blacktriangleleft$}
\newcommand{\rproof}{$\blacktriangleright$}
\newcommand{\norm}[1]{\left|\left|#1\right|\right|}
\newcommand{\diag}{\mathop{\mathrm{diag}}} 
\newcommand{\mat}[2][ccccccccccccccccccccccccc]{\left[
    \begin{array}{#1}
      #2\\
    \end{array}
  \right]}
\newcommand{\Eq}[1]{Eq. (\ref{eq:#1})}
\newcommand{\Fact}[1]{(\ref{fact:#1})}
\newcommand{\Fig}[1]{Figure \ref{fig:#1}}
\newcommand{\Empty}{\text{Empty}}
\newcommand{\nil}{\text{nil}}
\newcommand{\fun}[1]{\emph{\textcolor{blue}{#1}}}
\newcommand{\E}[1]{\textbf{E}[[#1]]}
\newcommand{\C}[1]{\textbf{C}[[#1]]}
\newcommand{\B}[1]{\textbf{B}[[#1]]}
\newcommand{\Prog}[1]{\textbf{Prog}[[#1]]}
\newcommand{\update}{\textcolor{blue}{\text{update}}}
\newcommand{\access}{\textcolor{blue}{\text{access}}}
\newcommand{\plus}{\textcolor{blue}{\text{plus}}}
\newcommand{\result}{\text{result}}
\newcommand{\letin}[2]{\textcolor{blue}{\text{let}}\,#1 = #2\,\textcolor{blue}{\text{in}}}
\newcommand{\If}{\textcolor{blue}{\text{if}}}
\newcommand{\Then}{\textcolor{blue}{\text{then}}}
\newcommand{\Else}{\textcolor{blue}{\text{else}}}
\newcommand{\equals}{\textcolor{blue}{\text{equals}}}
\newcommand{\minus}{\textcolor{blue}{\text{minus}}}




\mathchardef\mhyphen="2D
\usepackage[pagebackref=true,breaklinks=true,letterpaper=true,colorlinks,bookmarks=false]{hyperref}

\newcommand*{\Comment}[1]{\hfill\makebox[3.0cm][l]{#1}}%
% Copied from http://en.wikibooks.org/wiki/LaTeX/Packages/Listings
\lstset{ %
  language=C++,                % the language of the code
  basicstyle=\small,           % the size of the fonts that are used for the code
  columns=fullflexible,
  numbers=left,                   % where to put the line-numbers
  numberstyle=\tiny\color{black},  % the style that is used for the line-numbers
  stepnumber=1,                   % the step between two line-numbers. If it's 1, each line 
                                  % will be numbered
  numbersep=10pt,                  % how far the line-numbers are from the code
  backgroundcolor=\color{white},      % choose the background color. You must add \usepackage{color}
  showspaces=false,               % show spaces adding particular underscores
  showstringspaces=false,         % underline spaces within strings
  showtabs=false,                 % show tabs within strings adding particular underscores
  frame=trBL,                   % adds a frame around the code
  frameround = fttt,
  % rulecolor=\color{black},        % if not set, the frame-color may be changed on line-breaks within not-black text (e.g. commens (green here))
  tabsize=2,                      % sets default tabsize to 2 spaces
  captionpos=b,                   % sets the caption-position to bottom
  breaklines=true,                % sets automatic line breaking
  breakatwhitespace=false,        % sets if automatic breaks should only happen at whitespace
  title=\lstname,                   % show the filename of files included with \lstinputlisting;
                                  % also try caption instead of title
  keywordstyle=\color{blue},          % keyword style
  commentstyle=\color{dkgreen},       % comment style
  stringstyle=\color{mauve},         % string literal style
  escapeinside={\%*}{*)},            % if you want to add a comment within your code
  morekeywords={*,cases,onError}               % if you want to add more keywords to the set
}

\begin{document}
\maketitle

\begin{enumerate}
\item \textbf{Problem} For a given set of slots (each slot can be of 1
  hole, 2 holes or 3 holes), and a given set of jewels (each jewel
  occupies 1 hole, 2 holes or 3 holes, and are called size-1 jewel,
  size-2 jewel and size-3 jewel respectively), find the set of
  combination of jewels that fit into the set of slots. Each Jewels
  can be used multiple times.
\item \textbf{Model} Assign each jewel an unique id, and ensure that
  all size-3 jewel ids > all size-2 jewel ids > all size-1 jewel ids
  >= the dummy size-1 jewel (represent an empty embedding) ids.
  \begin{itemize}
  \item Define $C_1$ as the set of all the 1-hole jewels. Define
    $C_2$, $C_3$ analogical to $C_1$.
  \item Represent each jewel combination as an ordered collection of
    their ids. For example, $(1, 3, 3, 4)$ is a jewel combination of 4
    jewels, and $\{(1, 1), (1, 2), (5)\}$ is a set of jewel
    combinations.
  \item Define the operator $+$ over the \textbf{set of} set of jewel
    combinations: $A + B = A \cup B$, and $+$ is valid if and only if
    $A \cap B = \emptyset$.
  \item Define the operator $\times$ over the \textbf{set of} set of
    jewel combinations:
    \[
    A \times B = \left\{ (a_1, \cdots, a_m, b_1, \cdots, b_n) \mid  (a_1, \cdots, a_m) \in A \text{ and } (b_1, \cdots, b_n) \in B
      \text{ and } a_m \leq b_1 \right\}
    \]
    where $a_1, \cdots$ are all jewel ids.
  \item Define $f(i, j, k)$ to be the set of all the jewel
    combinations that fits into $i$ 1-hole, $j$ 2-hole and $k$ 3-hole
    slots. Define $f(i, \overline{j}, k)$ to be similar to $f(i, j,
    k)$, but the $j$ 2-hole slots can only hold jewels that occupy
    exactly 2 holes (size-2 jewels). The same goes for $i$, $k$ when
    there is a bar over each one of them.

    Example:
    \begin{itemize}
    \item[(1)] $f(i,j,\overline{k}) = $ the set of all jewel combinations
      that fits into $i$ 1-hole, $j$ 2-hole and $k$ 3-hole slots,
      where the $k$ 3-hole slots can only be occupied by size-3
      jewels.
    \item[(2)] $f(i,\overline{j},\overline{k}) = $ the set of all
      jewel combinations that fits into $i$ 1-hole, $j$ 2-hole and $k$
      3-hole slots, where the $k$ 3-hole slots can only be occupied by
      size-3 jewels and the $j$ 2-hole slots can only be occupied by
      size-2 jewels.
    \end{itemize}
  \end{itemize}


\item \textbf{Analysis}
  Let us start from the simplest case. 
  \begin{itemize}
  \item $f(i, 0, 0)$. 


    It is easy to find that 
    \begin{equation}
      \label{eq:f100}
      f(1, 0, 0) = C_1
    \end{equation}
    and also, force 1-hole slot to take size-1 jewel does not make a
    difference:
    \begin{equation}
      f(\overline{i}, 0, 0) = f(i, 0, 0)
    \end{equation}
    Besides, to iteratively generate $f(i, 0, 0)$, we have
    \begin{equation}
      f(i, 0, 0) = f(i-1, 0, 0) \times C_1 \quad (i > 1)
    \end{equation}


  \item $f(i, j, 0)$.
    
    Now we introduce 2-hole slots. For $f(i, j, 0)$, it's easy to find
    out that there are only two cases:
    \begin{itemize}
    \item[(1)] All the $j$ 2-hole slots take size-2 jewels. This is
      exactly $f(i, \overline{j}, 0)$.
    \item[(2)] At least one of the $j$ 2-hole slots is taking size-1
      jewels. This is exactly $f(i + 2, j-1, 0)$.
    \end{itemize}

    Note that the above two cases are disjoint. Which makes it easy to
    find the iterative formula for $f(i, j, 0)$. Basically, when $j >
    0$,
    
    \begin{align}
      f(i, j, 0) &= f(i, \overline{j}, 0) + f(i + 2, j -1 , 0) \\
      &= f(i, 0, 0) \times f(0, \overline{j}, 0) + f(i + 2, j - 1, 0) 
    \end{align}
    
    This shows that we also need to maintain the state of $f(0,
    \overline{j}, 0)$ in the implementation, namely
    \begin{equation}
      \label{eq:f0j0}
      f(0, \overline{j}, 0) = f(0, \overline{j - 1}, 0) \times C_2
    \end{equation}

  \item $(i, j, k)$.

    This is the most complicated cases, at least it seems to
    be. Though it can still be divided into two cases:
    \begin{itemize}
    \item[(1)] All the $k$ 3-hole slots take size-3 jewels. This is
      exactly $f(i, j, \overline{k})$.
    \item[(2)] At least one of the $k$ 3-hole slots is taking size-1
      or size-2 jewels. This is actually $f(i + 1, j + 1, k -
      1)$. \textbf{Note that} we do not need to count $f(i + 3, j, k
      -1)$ since $f(i + 3, j, k - 1) \in f(i + 1, j + 1, k -1)$!
    \end{itemize}

    The bottom line is that when $k > 0$,
    \begin{align}
      f(i, j, k) &= f(i, j, \overline{k}) + f(i + 1, j + 1, k - 1) \\
      &= f(i, j, 0) \times f(0, 0, \overline{k}) + f(i + 1, j + 1, k - 1)
    \end{align}

    This requires maintaining states for $f(0, 0, \overline{k})$ via

    \begin{equation}
      \label{eq:f00k}
      f(0, 0, \overline{k}) = f(0, 0, \overline{k-1}) \times C_3
    \end{equation}
  \end{itemize}

  With all the equations, we are now finally able to calculate each
  $f(i, j, k)$ in a mnemonic lazy-way.
\end{enumerate}


\end{document}